%!TEX root = ../DDE-Project-Master.tex

\section{Preliminaries}

\subsection{DDEs covered by the proposed approach} We consider systems of nonlinear DDEs involving multiple discrete or distributed delays, either in the linear term or in the nonlinearity. Such DDEs can be put into the following form: 
\bea \label{Eq_DDE_general}
\frac{\d \boldsymbol{x}(t)}{\d t}  = \boldsymbol{A} \boldsymbol{x}(t) & + \sum_{i=1}^p \boldsymbol{B}_i  \boldsymbol{x}(t-\tau_i) + \sum_{i=1}^p \boldsymbol{C}_i  \int^t_{t-\tau_i} \boldsymbol{x}(s) \d s \\
& +  \boldsymbol{F} \left(t, \boldsymbol{x}(t), \boldsymbol{x}(t-\tau_1), \cdots, \boldsymbol{x}(t-\tau_p), \int^t_{t-\tau_1} \boldsymbol{x}(s) \d s, \cdots, \int^t_{t-\tau_p} \boldsymbol{x}(s) \d s\right),
\eea
where the unknown function $\boldsymbol{x}$ is a $d$-dimensional vector; $p$ is a positive integer, representing the total number of delays; the $\tau_i$'s are distinctive positive scalars arranged in ascending order; $\boldsymbol{A}$, $\boldsymbol{B}_i$, and $\boldsymbol{C}_i$ ($1\le i \le p$) are given $d\times d$ matrices; and $\boldsymbol{F}\colon \mathbb{R}^{2+2p} \rightarrow \mathbb{R}^d$ is a given continuous vector function.

In order to simplify the presentation, we first articulate our main contribution in a simple setting of a scalar DDE with a single discrete delay $\tau>0$:
%\bea\label{eq:dde}
%    \frac{\d x(t)}{\d t} &= ax(t) + bx(t-\tau) + F(x(t-\tau)), & t&> 0 \\
%    x(t) &= \varphi(t), & t&\in[-\tau, 0],
%\eea
\be\label{eq:dde}
    \frac{\d x(t)}{\d t} = ax(t) + bx(t-\tau) + F(x(t-\tau)), 
\ee
where \(a,b\in\R\), and \(F:\R \to \R\) is a given scalar function. Results for the general case of \eqref{Eq_DDE_general} is provided afterward in Section~\ref{Sect_convergence_system_case}. 


{\alert************************************
\bi
\item Explain in a short paragraph the main difficult compared with the case dealt with in \cite{CGLW16}.
\item To cope with the difficulties, we restrict the initial data to $C^2$ functions. Refer to Section~\ref{Sect_convergence_system_case} for results about existence and regularity. 
\item Make sense of the variation of constants formula.
\ei

************************************
}


%\in L^2([-\tau, 0];\R)\), 


It is appropriate to formulate {\attn this problem} into the Hilbert space 
\be
    \mcH := L^2([-\tau, 0);\R) \times \R,
\ee
where the inner product is defined for \((f_1, \gamma_1),\, (f_2, \gamma_2) \in \mcH \), as:
\be \label{H_inner}
    \langle (f_1, \gamma_1), (f_2, \gamma_2) \rangle_{\mathcal{H}} := \frac1 \tau \int_{-\tau}^0 f_1(\theta)f_2(\theta)  \d \theta  + \gamma_1\gamma_2.
\ee
 Let us denote by \(x_t\) the time evolution of the history segments of a solution to \cref{eq:dde}, i.e., 
\be
    x_t(\theta) := x(t+\theta), \qquad t\geq0, \qquad \theta\in[-\tau, 0].
\ee
Then by introducing 
\be
    u(t) := (x_t, x_t(0)),
\ee
we can rewrite \cref{eq:dde} as the following abstract ODE on \(\mcH\):
\be \label{eq:abstract_ode}
    \frac{\d u}{\d t}  = \mcA u + \mcF (u).
\ee
The linear operator \(\mcA: D(\mcA) \to \mcH\) is defined by
\bea \label{def:mcA}
    \lbrack \mcA \Psi \rbrack (\theta) & := \begin{cases}
    {\displaystyle \frac{\d^+ \Psi^D}{\d \theta}}, &  \theta \in[-\tau, 0),  \vspace{0.5em}\\ 
    {\displaystyle a\Psi^S + b\Psi^D(-\tau)}, & \theta = 0,
    \end{cases}   
\eea
for any $\Psi = (\Psi^D, \Psi^S)$ that lives  in the domain,  $D(\mathcal{A})$, defined as
\be \label{D_of_A2}
    D(\mcA): = \left\{\Psi \in \mcH : \Psi^D \in H^1([-\tau, 0); \mathbb{R}^d), \lim_{\theta \to 0^-} \Psi^D(\theta) =\Psi^S \right\}.
\ee
The {\attn nonlinear operator \(\mcF:\mcH \to \mcH\)} is defined by 
\bea \label{mcF}
    [\mcF (\Psi) ](\theta) & := \begin{cases}
    0, &  \theta \in[-\tau, 0),   \vspace{0.4em}\\ 
    F \left( \Psi^D(-\tau) \right), & \theta = 0, 
    \end{cases}  \quad {\attn \Forall \Psi = (\Psi^D, \Psi^S) \in  \mcH}.
\eea
% This section should discuss how Koornwinder Polynomials are defined. It will also go over the properties of S_N that will be used later for proofs.
\subsection{Properties and Basic Results of Koornwinder Polynomials}

\TODO{add a discussion of Koornwinder polynomials here}

From \cite[Eq.~(2.1)]{Koornwinder}, the sequence of Koornwinder polynomials \(\{K_n\}\) can be built from the Legendre polynomials \(L_n\) by 
\be\label{eq:Pn}
    K_n(s) := -(1+s)\frac d {\d s} L_n(s)+ (n^2+n+1)L_n(s), \quad s\in[-1,1],\  n\in\Nzero.
\ee

Furthermore, we reproduce from \cite[Prop.~3.1]{CGLW16} some simple properties that \(\{K_n\}\) satisfy.
\bprop
The polynomial \(K_n\) defined in \eqref{eq:Pn} is of degree \(n\) and  admits the following expansion in terms of the Legendre polynomials:
\be\label{eq:Pn2}
    K_n(s) = - \sum_{j = 0}^{n-1} (2j+1)L_j(s) + (n^2 + 1) L_n(s), \qquad n \in \Nzero;
\ee
and the following normalization property holds:
\be
    K_n(1) = 1, \qquad n \in \Nzero.
\ee

Moreover, the sequence given by
\be
    \{\mathcal{K}_n := (K_n, K_n(1)) : n \in \Nzero\}
\ee 
forms an orthogonal basis of the  product space 
\be
    \mathcal{E} := L^2([-1,1); \R) \times  \R,
\ee 
where \(\mcE\) is endowed with the following inner product:
\be
    \langle (f, a), (g, b) \rangle_\mcE = \frac 1 2 \int_{-1}^1 f(s)g(s) \d s  + ab, \quad (f,a), (g, b) \in \mcE.
\ee

Finally, \(\left\{ \frac{\mcK_n}{\| \mcK_n \|_\mcE} \right\}\) forms a Hilbert basis of $\mcE$ where 
the norm \(\| \mcK_n \|_\mcE\) of \(\mcK_n\) induced by  \(\langle \cdot, \cdot \rangle_\mcE\)  possesses the following analytic expression:
\be \label{eq:Pn_norm}
    \|\mcK_n\|_\mcE = \sqrt{\frac{(n^2+1)((n+1)^2+1)}{2n+1}}, \qquad n \in \Nzero.
\ee
\eprop

Suppose that \(\Pi_N\) is the \(N\)-dimensional standard projection into \(\mbox{span}\{\mcK_n: n\leq N\} \subset \mcE\). It will be relevant to discuss when we have convergence of \([\Pi_N u]^D\) for \(u\in\mcE\). In particular, we will focus on uniform convergence. {\attn We define for \(f\in L^2([-1,1], \mathbb R)\) the following}:
\be
a_n(f) := \frac{2n + 1} 2 \int_{-1}^1 f(x)L_n(x)\d x.
\ee

\bprop\label{prop:uniform-conv}
Let \(f\in \mathcal C^2([-1,1];\mathbb R)\) and denote \(\psi = (f, f(0)) \in\mcE\). Then the series
\be\label{koorn-series}
    [\Pi_N\psi]^D = \sum_{n=0}^N \frac{\langle \psi, \mcK_n \rangle_\mcE}{\| \mcK_n \|^2_\mcE} K_n
\ee
converges uniformly to \(f\).
\eprop
See Appendix~\ref{Appendix_proofs} for a proof.

It will also be necessary to prove certain properties of the series of Koornwinder polynomials
\be\label{eq:SN}
    S_N(x) := \sum_{n=0}^N \frac{K_n}{\| \mcK_n \|_\mcE^2}, \quad N\in\Nzero,\  x\in[-1,1].
\ee
If we were to denote \(\psi = (0,1) \in L^2([-1,1)) \times \R\), then \(S_N\) would simply be the functional part of \(\Pi_N\psi\). {\attn The following lemma allows us to express \(S_N\) in terms of Legendre Polynomials. Now that we have this expression, we can prove the properties of \(S_N\) that will be useful when showing the main result.}

\bl \label{Lemma_S_N}
The functions \(S_N\) defined in \eqref{eq:SN} can be expressed as
\be
S_N(x) = \frac 1 {(N + 1)^2 +1} \sum_{n=0}^N (2n + 1) L_n, \quad x\in[-1,1].
\ee

Moreover,
\be\label{eq:uniform-bdd}
    |S_N(x)| < 1, \quad \forall N\in\Nzero,\ \forall x\in[-1,1],
\ee
and
\be\label{eq:pw-conv}
    \lim_{N\to\infty} S_N(x) = 0, \quad \forall x\in(-1,1).
\ee
\el
See Appendix~\ref{Appendix_proofs} for a proof.

\br
{\alert Make a comment about convergence at the two endpoints.}
\er


\subsection{The Space \(X\)}

We define the following inner product space with elements in 
\be
    X := \mathcal C([-\tau, 0) ; \mathbb R) \times \mathbb R
\ee
and the inner product defined by
\be
    (\Phi, \Psi)_X := \Phi^S\Psi^S + \frac 1 \tau (\Phi^D, \Psi^D)_{L^2([-\tau, 0)} + \Phi^D(-\tau)\Psi^D(-\tau), \quad \Phi,\Psi\in X.
\ee
It is relatively straight-forward to verify that \((\cdot, \cdot)_X\) is symmetric, bilinear, and positive definite and thus is an inner product. We will also make use of the norm \(\|\cdot\|_X\) induced from this inner product. Note that \(X\) is \textbf{not} a Banach space since Cauchy sequences might not converge in \(X\).